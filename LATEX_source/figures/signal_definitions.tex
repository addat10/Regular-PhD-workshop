\tikzstyle{block} = [draw, rectangle, 
    minimum height=3em, minimum width=6em]
\tikzstyle{sum} = [draw,  circle, node distance=1cm]
\tikzstyle{input} = [coordinate]
\tikzstyle{output} = [coordinate]
\tikzstyle{pinstyle} = [pin edge={to-,thin,black}]

% The block diagram code is probably more verbose than necessary
\begin{tikzpicture}[auto, node distance=2cm,>=latex']
    % We start by placing the blocks
    \node [block,xshift=-1cm] (controller) 
    {$\begin{bmatrix}
    -mI_d  & I_d \\
    LI_d    & -I_d
    \end{bmatrix}$};
    \node [block, right of=controller,xshift=2cm,yshift=0.6cm] (filter1) {$\left[e^{-2\alpha t}h(-t)\right]*p(t)$};
    \node [block, right of=controller,xshift=2cm,yshift=-0.6cm,minimum width=3.15cm] (filter2) {$\left[e^{-2\alpha t}h(t)\right]*q(t)$};
    \node [input, left=of controller.170,xshift=1cm] (input1) {};
    \node [input, left=of controller.190,xshift=1cm] (input2) {};
    \node [input, left=of filter1.west,xshift=1.5cm] (p1) {};
    \node [input, left=of filter2.west,xshift=1.5cm] (q1) {};
    % We draw an edge between the controller and system block to 
    % calculate the coordinate u. We need it to place the measurement block. 
    \node [output, right of=filter1] (output1) {};
    \node [output, right of=filter2] (output2) {};

    % Once the nodes are placed, connecting them is easy.
    \draw[->] (input1) -- node [name=ytilde] {$\Tilde{y}$}(controller.170);
    \draw[->] (input2) -- node [name=utilde] {$\Tilde{u}$}(controller.190);
    \draw[-] (controller.10) -| node [name=p] {}(p1);
    \draw[-] (controller.350) -| node [name=q] {}(q1);
    \draw[->] (p1) -- node [name=p] {$p$}(filter1.west);
    \draw[->] (q1) -- node [name=q] {$q$}(filter2.west);
    \draw[->] (filter1) -- node [name=w2tilde] {$w_2$}(output1);
    \draw[->] (filter2) -- node [name=w1tilde] {$w_1$}(output2);

\end{tikzpicture}
