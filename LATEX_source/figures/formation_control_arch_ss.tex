\tikzstyle{block} = [draw, rectangle, 
    minimum height=3em, minimum width=6em]
\tikzstyle{sum} = [draw,  circle, node distance=1cm]
\tikzstyle{input} = [coordinate]
\tikzstyle{output} = [coordinate]
\tikzstyle{pinstyle} = [pin edge={to-,thin,black}]

% The block diagram code is probably more verbose than necessary
%\begin{tikzpicture}[auto, node distance=2cm,>=latex']
%    % We start by placing the blocks
%    \node [input, name=input] {};
%    \node [sum, right of=input] (sum) {};
%    \node [block, right of=sum] (controller) {Controller};
%    \node [block, right of=controller, pin={[pinstyle]above:Disturbances},
%            node distance=3cm] (system) {System};
%    % We draw an edge between the controller and system block to 
%    % calculate the coordinate u. We need it to place the measurement block. 
%    \draw [->] (controller) -- node[name=u] {$u$} (system);
%    \node [output, right of=system] (output) {};
%    \node [block, below of=u] (measurements) {Measurements};
%
%    % Once the nodes are placed, connecting them is easy. 
%    \draw [draw,->] (input) -- node {$r$} (sum);
%    \draw [->] (sum) -- node {$e$} (controller);
%    \draw [->] (system) -- node [name=y] {$y$}(output);
%    \draw [->] (y) |- (measurements);
%    \draw [->] (measurements) -| node[pos=0.99] {$-$} 
%        node [near end] {$y_m$} (sum);
%\end{tikzpicture}

\begin{tikzpicture}[auto, node distance=2cm,>=latex']
% We start by placing the blocks
\node [input, name=input] {};
\node [sum,right of=input](sum_ref) {};
\node [block, right of=input,node distance=3cm](L) {$\mathcal{L}\otimes I_d$};
\node [sum, right of=L,node distance=2cm](sum) {};
\node [block, right of=sum,node distance=3cm,minimum height=2.5em,minimum width=3em](G) {$\left[\begin{array}{c|c}
\hat{A}_G(\rho)     &  \hat{B}_G(\rho)\\
\hline
\hat{C}_G(\rho)     &  \mathbf{0}
\end{array}\right]$};
\node [output, right of=G,node distance=3cm](output) {};
\node [output, right of=G,node distance=2.1cm](outmid) {};
\node [block, above of=G,node distance=2.5cm](u_psi) {$\begin{bmatrix}0\\ \nabla \psi(y_i)\\ 0\\ \vdots \end{bmatrix}$};
\node [output, below of=L,node distance=2cm](delta) {};


% Once the nodes are placed, connecting them is easy. 
\draw [draw,->] (input) -- node {$-{r}$} (sum_ref);
\draw [draw,->] (sum_ref) -- (L);
\draw [draw,->] (L) -- (sum);
\draw [draw,->] (u_psi) -| (sum);
\draw [draw,->] (sum) -- node {${u}$} (G);
\draw [draw,->] (G) -- node{${y}$} (output);
\draw [-] (outmid) |- (delta);
\draw [->] (outmid) |- (u_psi);
\draw [->] (delta) -| (sum_ref);
\end{tikzpicture}